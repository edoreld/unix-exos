%%%%%%%%%%%%%%%%%%%%%%%%%%%%%%%%%%%%%%%%%
% Structured General Purpose Assignment
% LaTeX Template
%
% This template has been downloaded from:
% http://www.latextemplates.com
%
% Original author:
% Ted Pavlic (http://www.tedpavlic.com)
%
% Note:
% The \lipsum[#] commands throughout this template generate dummy text
% to fill the template out. These commands should all be removed when 
% writing assignment content.
%
%%%%%%%%%%%%%%%%%%%%%%%%%%%%%%%%%%%%%%%%%

%----------------------------------------------------------------------------------------
%	PACKAGES AND OTHER DOCUMENT CONFIGURATIONS
%----------------------------------------------------------------------------------------

\documentclass{article}

\usepackage{fancyhdr} % Required for custom headers
\usepackage{lastpage} % Required to determine the last page for the footer
\usepackage{extramarks} % Required for headers and footers
\usepackage{graphicx} % Required to insert images
\usepackage{lipsum} % Used for inserting dummy 'Lorem ipsum' text into the template
\usepackage{listings}
\usepackage[utf8]{inputenc}
\usepackage[T1]{fontenc}% http://ctan.org/pkg/fontenc
\usepackage{upquote}
\lstset{upquote=true}


% Margins
\topmargin=-0.45in
\evensidemargin=0in
\oddsidemargin=0in
\textwidth=6.5in
\textheight=9.0in
\headsep=0.25in 

\linespread{1.1} % Line spacing

% Set up the header and footer
\pagestyle{fancy}
\lhead{\hmwkAuthorName} % Top left header
\chead{} % Top center header
\rhead{\hmwkClass\ (\hmwkTitle)} % Top right header
\lfoot{\lastxmark} % Bottom left footer
\cfoot{} % Bottom center footer
\rfoot{Page\ \thepage\ of\ \pageref{LastPage}} % Bottom right footer
\renewcommand\headrulewidth{0.4pt} % Size of the header rule
\renewcommand\footrulewidth{0.4pt} % Size of the footer rule

\setlength\parindent{0pt} % Removes all indentation from paragraphs

%----------------------------------------------------------------------------------------
%	DOCUMENT STRUCTURE COMMANDS
%	Skip this unless you know what you're doing
%----------------------------------------------------------------------------------------

% Header and footer for when a page split occurs within a problem environment
\newcommand{\enterProblemHeader}[1]{
\nobreak\extramarks{#1}{#1 continued on next page\ldots}\nobreak
\nobreak\extramarks{#1 (continued)}{#1 continued on next page\ldots}\nobreak
}

% Header and footer for when a page split occurs between problem environments
\newcommand{\exitProblemHeader}[1]{
\nobreak\extramarks{#1 (continued)}{#1 continued on next page\ldots}\nobreak
\nobreak\extramarks{#1}{}\nobreak
}

\setcounter{secnumdepth}{0} % Removes default section numbers
\newcounter{homeworkProblemCounter} % Creates a counter to keep track of the number of problems

\newcommand{\homeworkProblemName}{}
\newenvironment{homeworkProblem}[1][Problemme \arabic{homeworkProblemCounter}]{ % Makes a new environment called homeworkProblem which takes 1 argument (custom name) but the default is "Problem #"
\stepcounter{homeworkProblemCounter} % Increase counter for number of problems
\renewcommand{\homeworkProblemName}{#1} % Assign \homeworkProblemName the name of the problem
\section{\homeworkProblemName} % Make a section in the document with the custom problem count
\enterProblemHeader{\homeworkProblemName} % Header and footer within the environment
}{
\exitProblemHeader{\homeworkProblemName} % Header and footer after the environment
}

\newcommand{\problemAnswer}[1]{ % Defines the problem answer command with the content as the only argument
\noindent\framebox[\columnwidth][c]{\begin{minipage}{0.98\columnwidth}#1\end{minipage}} % Makes the box around the problem answer and puts the content inside
}

\newcommand{\homeworkSectionName}{}
\newenvironment{homeworkSection}[1]{ % New environment for sections within homework problems, takes 1 argument - the name of the section
\renewcommand{\homeworkSectionName}{#1} % Assign \homeworkSectionName to the name of the section from the environment argument
\subsection{\homeworkSectionName} % Make a subsection with the custom name of the subsection
\enterProblemHeader{\homeworkProblemName\ [\homeworkSectionName]} % Header and footer within the environment
}{
\enterProblemHeader{\homeworkProblemName} % Header and footer after the environment
}
   
%----------------------------------------------------------------------------------------
%	NAME AND CLASS SECTION
%----------------------------------------------------------------------------------------

\newcommand{\hmwkTitle}{États\ des\ processus} % Assignment title
\newcommand{\hmwkDueDate}{22 Mars 2017 avant 13:45} % Due date
\newcommand{\hmwkClass}{Utilisation\ et\ Programmation UNIX} % Course/class
\newcommand{\hmwkClassTime}{} % Class/lecture time
\newcommand{\hmwkClassInstructor}{Arhodakis Georgios} % Teacher/lecturer
\newcommand{\hmwkAuthorName}{Francisco Javier Martínez Lago} % Your name

%----------------------------------------------------------------------------------------
%	TITLE PAGE
%----------------------------------------------------------------------------------------

\title{
\vspace{2in}
\textmd{\textbf{\hmwkClass:\ \hmwkTitle}}\\
\normalsize\vspace{0.1in}\small{À rendre\ le\ \hmwkDueDate}\\
\vspace{0.1in}\large{\textit{\hmwkClassInstructor\ \hmwkClassTime}}
\vspace{3in}
}

\author{\textbf{\hmwkAuthorName}}
\date{} % Insert date here if you want it to appear below your name

%----------------------------------------------------------------------------------------

\begin{document}

\maketitle

%----------------------------------------------------------------------------------------
%	PROBLEM 1
%----------------------------------------------------------------------------------------

% To have just one problem per page, simply put a \clearpage after each problem

\begin{homeworkProblem}[1. Détail de la colonne S du processus top] % Custom section title

\problemAnswer{ % Answer

La colonne S du processus top représente le Process Status (État d'un processus). Un processus sera en tout moment dans un état parmi les cinq états possibles : \emph{running} (R), \emph{uninterruptible sleep} (D), \emph{sleeping} (S), \emph{traced} ou \emph{stopped} (T) et \emph{zombie} (Z).
	
\begin{itemize}
	\item Un processus ayant l'état \textbf{running} est en cours d'exécution. Ça veut dire qu'il est actif et en train d'utiliser la CPU du système.
	\item L'état \textbf{sleeping} est l'état des processus qui sont suspendus : il n'utilisent pas la CPU du système et il peuvent être interrompus en utilisant de signales. Ce sont des processus qui attendent qu'un événement finisse pour se réveiller.
	\item L'état \textbf{uninterruptible sleep} correspond à des processus qui sont suspendus et qui ne peuvent pas être interrompus. Normalement il sont en train d'attendre une opération d'entrée-sortie du disque. 
	\item L'état \textbf{traced} ou \textbf{stopped} est l'état d'un processus qui a été arrêté par une signale, usuellement par SIGSTOP ou SIGTSTP. Il peut être résume en lui envoyant la signale SIGCONT.
	\item On appelle \textbf{zombie} un processus dont son exécution est finie avec un code spécifique - le exit code. Le processus attend seulement que son parent collecte ce code. 
\end{itemize} 
	}


\end{homeworkProblem}

%----------------------------------------------------------------------------------------
%	PROBLEM 2
%----------------------------------------------------------------------------------------

\begin{homeworkProblem}[2. Quelles sont les paramètres de ps pour avoir uniquement les processus d'un état ?] % Custom section title

\end{homeworkProblem}

\problemAnswer{ %Answer

Une solution possible à ce problème en utilisant la commande awk :  

\begin{lstlisting}[language=bash]^^J
	ps -aux | awk \{'if (substr($8,1,1) ==  "S") print $0'\}
\end{lstlisting}

Dans l'exemple précedent on utilise un tube pour diriger la sortie de ps vers awk. awk va comparer la première lettre de la colonne 8 - où il y a les états - avec une chaîne de caractères correspondant au lettre de l'état qu'on veut. Pour chaque ligne où cette comparaison est vraie, il va l'imprimer. \newline

Bien sûr il faut remplacer à chaque fois le lettre S pour le lettre correspondant à l'état qu'on veut. Pour éviter de devoir à chaque fois taper la commande avec la lettre, on peut envisager de créer une fonction.

	Une manière de le faire :
	
	\begin{lstlisting}[language=bash]^^J	
	ps_for_stat() \{  
	\end{lstlisting}
	
	\begin{lstlisting}[language=bash]^^J	
	ps -aux | awk \{'if (substr($8,1,1) ==  "'"$1"'") print $0'\}
	\end{lstlisting}
	
	\begin{lstlisting}[language=bash]^^J	
	\}  
	\end{lstlisting}
	
	On pourrait par exemple mettre ce code dans ~/.bashrc et ajouter en suite:
	
	\begin{lstlisting}[language=bash]^^J	
	import -f ps_for_stat
	\end{lstlisting}	
	
	Finalement, on pourrait l'utiliser dans une nouvelle terminale 
	
	\begin{lstlisting}[language=bash]^^J	
		ps_for_stat S
	\end{lstlisting}
	
	Ce qui nous donnera - au moins sur Ubuntu 16.04 - une liste de processus avec l'état correspondant au paramètre passé par ligne de commande. 	

}

\end{document}
